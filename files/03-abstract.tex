%!TEX root = ../main.tex

\begin{abstract}
\thispagestyle{plain}
\setcounter{page}{5}

Chills are a fleeting, pleasurable bodily sensation, sometimes accompanied by piloerection, experienced when listening to specific musical passages. They are often used as an indicator of aesthetic and emotional responses to music, because they are considered to be pleasurable, widespread, memorable, and observable. However, research on chills suffers from theoretical and practical limitations. Notably, there are significant shortcomings in the available literature regarding research design, adequacy of experimental variables, and empirical measures of chills. This thesis reviews the suitability of using chills in musical aesthetics and emotion research, and details the construction of several large-scale datasets of pieces of music that elicit chills, in order to solve ongoing issues of reproducibility, generalisability, and ecological validity. These datasets are used in a range of behavioural and computational studies, applying paradigms that address current gaps in research on chills, and particularly the widely hypothesised role of musical expectation and associated factors, which lacks any form of direct empirical verification. More specifically, this thesis presents 1) a systematic review of the literature on chills in music, 2) a longitudinal study using a combination of self-reported and objective measures of chills to assess the roles of musical content, stylistic preference, and familiarity, 3) a corpus analysis investigating conflicting effects of perceived valence on chills, and 4) a large-scale computational study modelling the onsets of chills from acoustic and syntactic properties, resulting in a system with the potential to predict if and when chills might occur in a piece of music. While chills have often been described as idiosyncratic, this thesis demonstrates strong associations between chills and expectation, providing further clarity on competing psychological theories about the origins of chills, and of music appreciation in general.

\end{abstract}