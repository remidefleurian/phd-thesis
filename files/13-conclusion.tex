%!TEX root = ../main.tex

\chapter{Conclusion}
\label{ch:6}

\section{Overview}

In this thesis, we presented a body of work aiming to investigate the relationships between MECs, musical expectation, and affective and aesthetic responses, considering research questions about what MECs are, what causes them, and when they occur.

In Chapter \ref{ch:2}, we conducted a systematic review of the literature on MECs, allowing us to define them as a fleeting, pleasurable bodily sensation, sometimes accompanied by goosebumps, experienced when listening to specific musical passages. In the review, we integrated theoretical and empirical findings to reveal that MECs are associated with physiological changes and increased arousal, and recruit brain structures and systems relevant to emotion, reward, and motivation. We identified that they can be caused by a set of acoustic, musical, and emotional elicitors, and are influenced by personality differences such as openness to experience. We provided a preliminary theoretical model that allows for different psychological pathways for the experience of MECs, if not different types of MECs, relying on complex interactions between listener, context, elicitors, psychological and evolutionary mechanisms, and response attributes. We provided a dataset of pieces of music known to cause MECs, and a set of open issues, hypotheses, and methodological recommendations, which motivated the research presented in the following chapters. Notably, we highlighted that further evidence was needed on the relationships between MECs and piloerection, pleasure, familiarity, stylistic preference, and musical expectation, and that causal approaches and the use of naturalistic listening experiences should be emphasised in future research.

In Chapter \ref{ch:3}, we followed these recommendations by investigating the relationships between MECs, piloerection, pleasure, musical content, stylistic preference, familiarity, and liking in a controlled, longitudinal experiment using existing pieces of music in as naturalistic a listening context as possible. Musical content, stylistic preference, and familiarity were systematically manipulated, allowing us to identify robust effects of stylistic preference on MECs, piloerection, and pleasure, and revealing that much of the variance in liking pieces of music was accounted for by all the other factors combined. There were fundamental differences between MECs and piloerection, with MECs being reported far more often than piloerection was detected, familiarity having opposite effects on these two responses, and the responses themselves being uncorrelated. However, piloerection significantly overlapped with MECs and pleasure, which suggests that, while related, piloerection might only be present for a distinct type of MECs, or that it requires MECs to exceed an intensity threshold. Effect sizes were small for all the identified effects, and since no effect of musical content on MECs was detected, we hypothesised that more powerful approaches were needed to identify such an effect.

In Chapter \ref{ch:4}, we implemented such an approach by conducting a corpus analysis using computational methods. We compared track-level audio features between tracks taken from the dataset provided in Chapter \ref{ch:2} and several sets of control tracks, algorithmically matched by artist, duration, and popularity, to investigate the effect of expressed valence on MECs. We identified that tracks known to elicit MECs were sadder than control tracks, and that they also tended to be slower, less intense, and more instrumental. Moreover, that effect of valence on MECs differed depending on the audio characteristics of the tracks taken from the dataset, suggesting the possibility of there being different causes of MECs for different types of music. Overall, this study provided further evidence for an effect of valence on MECs, and demonstrated that computational methods are well-suited to the study of MECs, enabling the findings that music that causes MECs differed in musical content from other music. However, track-level features were used, which are inadequate for an in-depth exploration of local elicitors of MECs, which occur transiently at particular points while listening to particular pieces of music. This motivated further, more thorough computational work in the following chapter.

In Chapter \ref{ch:5}, we conducted a computational analysis aimed at modelling the onset of MECs based on auditory and musical characteristics. We extracted a wide range of features and labels of onsets of MECs from the results of a survey study initiated in Chapter \ref{ch:3}, to investigate the effects of known acoustic and musical elicitors of MECs, as well as the often hypothesised effect of musical expectation on the occurrence of MECs. We ran a series of permutation tests to assess the local behaviour of each feature, and trained a series of models to evaluate how well they could predict onsets of MECs, and which features were most important in driving these predictions. This process resulted in a systematic, large-scale replication of all the effects of the elicitors of MECs that were included in the analysis. In addition, onsets of MECs could be predicted better than chance, and melodic expectation improved model performance and was the best predictor of MECs, with MECs being more likely in uncertain melodic contexts including a unexpected event followed by more expected events. The differences between feature importance and local behaviour for some features allowed us to identify that many acoustic elicitors of MECs might be best considered as necessary but not specifically predictive of experiences of MECs.

\section{Recurring themes}

There were four recurring themes that permeated the work presented in this thesis. First, we highlighted theoretical and practical limitations in prior research on MECs, with issues in terms of research design, adequacy of measures of MECs, reproducibility, generalisability, and ecological validity. We have provided contributions to addressing these issues, by conducting a systematic review of the diverse and fast-growing body of research on MECs, developing a preliminary theoretical model of MECs that provides a robust framework for future hypothesis-driven research, and providing a set of open questions, hypotheses, suggested approaches, and methodological recommendations for future research. We also compiled publicly available datasets to improve reproducibility and provide more representative data, made use of existing pieces in music in causal (and arguably double-blinded in Chapter \ref{ch:3}) or highly controlled hypothesis-driven studies, explored the use of computational methods which allowed improvements in generalisability and statistical power, and demonstrated the validity of a stimulus-matching paradigm over several studies.

Second, we hypothesised the presence of different psychological pathways for the experience of MECs. While the evidence we gathered is not sufficient to confirm all the predictions arising from the preliminary model of MECs presented in Chapter \ref{ch:2}, since that model represents a much larger research agenda than it was possible to cover empirically in this thesis, we did provide empirical support for several of its components, and were not able to refute any of the predictions made by the model. Notably, we confirmed the effects of many acoustic and musical elicitors of MECs, as well as an effect of valence as an emotional elicitor. We identified that the psychological mechanism of musical expectation was a strong predictor of MECs, which potentially provided more explanatory power than a hypothesised involvement of brain stem reflex. This result is particularly significant, because a relationship between expectation and MECs has been postulated continuously and very prominently in the literature since 1991, but there has been no convincing empirical evidence for a relationship to date. Finally, given the effect sizes we observed in behavioural and computational studies, we suspect that there is a ceiling to the effects of acoustic and musical elicitors on MECs, and that emotional elicitors such as meaning or emotionality might be at the origin of most experiences of MECs, or at least more influential, widespread, and consistent elicitors of MECs than acoustic and musical elicitors, through the process of being moved \parencite[for recent evidence of a strong relationship between MECs and being moved, see][]{vuoskoski2022}.

Third, we discussed the possibility that MECs are a collection of phenomenologically and psychologically distinct experiences, as identified in prior research which is discussed in Chapter \ref{ch:2}. Again, while we did not explicitly investigate this research question, and are therefore not able to provide conclusive evidence, our findings certainly provided some degree of support for this hypothesis. In fact, all the studies we conducted resulted in the identification of diverging patterns in the experience of MECs, such as fundamental differences between MECs and piloerection in Chapter \ref{ch:3}, differences in the effect of valence on MECs depending on stimulus-driven properties in Chapter \ref{ch:4}, or low predictive power of acoustic and musical elicitors of MECs in Chapter \ref{ch:5}. The presence of distinct types of MECs has received further support in the recent literature \parencite{bannister2020c,bannister2021}, and we consider this research question crucial to future work on MECs.

Lastly, we mentioned the lack of clarity in the relationship between MECs and emotional and aesthetic responses. In Chapter \ref{ch:2}, we characterised MECs as a pleasurable, though not essential component of emotional and aesthetic experiences. In Chapter \ref{ch:3}, our findings suggested that MECs overlapped significantly but not exclusively with pleasure, and that MECs, in combination with other factors such as stylistic preference and familiarity, could predict a large amount of the variance in music preference. In Chapter \ref{ch:5}, we identified that musical expectation was partially predictive of onsets of MECs---a result which can be placed in the context of prior research revealing that violations of musical expectation can induce emotional and aesthetic responses (see Chapter \ref{ch:1}). These results provide further justification for the recommendations we made in Chapter \ref{ch:2} that MECs should not be conflated with peak pleasure, and that while they can form a part of emotional and aesthetic responses to music, they should not be used as the sole indicator of such responses. Interestingly, this view is supported by recent evidence that, following administration of an opioid antagonist, experiences of MECs were characterised by no changes in self-reports of pleasure, but decreased pupil diameter \parencite{laeng2021}, therefore suggesting that the removal of a physiological component of MECs had no effect on experienced pleasure.

\section{Limitations and future work}

We opted to focus on MECs in the present research. However, it is important to keep in mind that pleasurable chills can also occur when presented with other forms of art. While some previous research has investigated such responses (see Chapter \ref{ch:2}), notably in comparative evaluations of emotional elicitors of chills, it is currently unknown whether or not the hypothesised psychological pathways for distinct experiences of MECs would apply to other types of art-elicited chills.

Many decisions about study design were taken in order to improve ecological validity, notably through the use of naturalistic stimuli, and to gather as much evidence as possible, by opting for interpretable models at the expense of predictive performance (although interpretability was also affected by the lack of perceptual validation of most of the features extracted for the analyses in Chapters \ref{ch:4} and \ref{ch:5}). We believe both of these decisions contributed to the small effect sizes observed throughout the present work, along with the impossibility of producing a comprehensive ground truth and the efforts made to provide rigorous control conditions. We suspect that complementary findings could be obtained from studies recruiting more powerful modelling approaches.

In terms of generalisability, while the datasets used in our research featured music from different genres and cultures, they were still mostly comprised of Western music. Relatedly, the behavioural study conducted in Chapter \ref{ch:3} suffered from the same pitfalls as many other psychology experiments with regards to sample representativeness. Cross-lab work involving online methods, as implemented by \textcite{jacoby2021}, is expensive in terms of time and resources, but provides an unparalleled opportunity to bring about generalisable cross-cultural findings, from which research on MECs could certainly benefit.

Finally, while establishing causality was a strong motivation for the present work, we generally focused our modelling efforts on maximising interpretability in order to generate novel findings about elicitors of MECs and demonstrate the suitability of computational approaches in the study of MECs. Empirical verification of the predictions made by these models is a necessary next step in order to gain confidence in such findings, and could be complemented by experiments seeking to manipulate stimuli in order to provide causal evidence for the effects of the identified psychological mechanisms underlying the experience of MECs.

Overall, while MECs are inherently subjective, hard to define and measure, and subject to complex interactions between listener, context, and music, they represent a fascinating opportunity to better understand why and how people appreciate music.
