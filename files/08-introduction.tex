%!TEX root = ../main.tex

\chapter{Introduction}
\label{ch:1}

\section{Empirical aesthetics}

Music is a human universal \parencite{mehr2019, savage2015} and is one of the most commonly reported sources of emotional pleasure \parencite{dube2003}. Yet, despite the prevalence of musical behaviours across cultures, the nature of the relationship between music, pleasure, and emotion is poorly understood. 

Empirical aesthetics were one of the very first topics of interest in experimental psychology. Early work by Gustav Fechner, Hermann von Helmholtz, and Wilhelm Wundt looked into the effect of stimulus complexity \parencite{wundt1863}, relationships of musical tones \parencite{helmholtz1863}, and learned associations \parencite{fechner1876} on aesthetic experience, and inspired a long tradition of psychological research into what elicits pleasurable responses to art \parencite[for a review, see][]{huron2016}. 

One notable contribution to the field comes from \textcite{berlyne1971}, who further investigated the relationship between complexity, arousal and pleasure, leading to the hypothesis that there exists an inverted-U relationship between hedonic value and arousal potential, in which arousal is influenced by stimulus properties such as complexity and familiarity, and pleasure is higher for moderate degrees of arousal, and lower for both high and low degrees of arousal. This hypothesis has been extensively tested in various artistic domains, including music, for which it has received relatively strong \parencite[e.g][]{heyduk1975, north1995, vitz1966}, though not unanimous \parencite[e.g.][]{orr2005, smith1990} empirical support \parencite[for reviews, see][]{chmiel2017, hargreaves2010, orr2005}.

The methods used in such research have become problematic in a few ways. First, they distinguish aesthetic judgement from emotional response \parencite{orr2005} and place the focus on preference, when more comprehensive accounts suggest a complex relationship between aesthetic response, emotion, and liking \parencite{juslin2010}, involving social, emotional, and cognitive components \parencite{konecni1979}, and depending on a reciprocal interplay between listener, context, and music \parencite{gabrielsson2001b, hargreaves2012}. As a result, while a focus on the relationship between preference and stimulus properties might enable convenient experimental approaches, it is unlikely to reflect the broad nature of affective and aesthetic responses and the psychological mechanisms that underlie them \parencite{huron2016, juslin2016}.

Second, research designs in empirical aesthetics have historically relied on tightly controlled experimental procedures in laboratory environments. While there are many benefits to this approach, it encourages the use of stimuli which only approximate music, in short-lived situations which are unlikely to induce fully fledged emotional and aesthetic responses. Ecological validity often comes at a cost, but when it comes to empirical aesthetics, it has been recommended to make use of more naturalistic listening experiences \parencite{hargreaves2010, hodges2016}, in addition to considering the possibility of employing longitudinal designs to study the development of responses to these experiences over time \parencite{greasley2016}.

Third, Berlyne's (\citeyear{berlyne1971}) and subsequent approaches do not always emphasise the temporal nature of music, and tend to consider musical stimuli as a whole instead. Music being a phenomenon that unfolds in time, dynamic fluctuations in affective and aesthetic responses should be taken into consideration \parencite{huron2010, mcdermott2012}. \textcite{madsen1993}, for instance, used continuous self-report methods, and found that the trajectories of aesthetic ratings are highly consistent within a piece of music for individual listeners, highlighting the need for the study of the underlying factors at the origin of these consistent patterns.

Lastly, there remain substantial challenges because of the lack of objective experimental variables with the potential to capture subjective experiences of pleasure and emotion. Self-report measures and physiological responses are traditionally used, but both have disadvantages in terms of susceptibility to biases, reliability, and specificity \parencite{juslin2016, larsen2008, orne1962, panksepp2002, zentner2010}. 

\section{Musical expectation}

Aesthetic experience depends on many factors. One prominent psychological phenomenon thought to be particularly relevant to empirical aesthetics is musical expectation. Research on the topic benefits from a long-standing theoretical background initiated by Eduard Hanslick (\citeyear{hanslick1854}), and reinforced by Leonard Meyer (\citeyear{meyer1956, meyer1957}) in parallel to the research discussed above on complexity, arousal, and pleasure. 

Musical expectation is based on the hypothesis that developing expectations follows a process of probabilistic learning of the statistical regularities in musical structure \parencite{pearce2018, saffran1999}. In other words, with exposure to a musical culture, listeners automatically and implicitly develop an internal model of the structure of a musical style through a process called statistical learning, which is then used, when listening to music, to form expectations about the possible continuations of the music through a process called probabilistic prediction \parencite{pearce2018}. These learned expectations can be violated, delayed, or confirmed, resulting in induced emotional and aesthetic responses \parencite{cheung2019, egermann2013, gold2019, huron2006, juslin2013, sauve2018, steinbeis2006}, possibly in order to drive learning to improve future predictions or otherwise optimise states of arousal \parencite{pearce2018}. 

Much research has been conducted on the topic of musical expectation. Behavioural methods have been used to demonstrate effects of melodic, harmonic, rhythmic or metrical expectation on recognition memory, music production, music perception, and music transcription \parencite[for a review, see][]{pearce2012}. Research using neuroscientific methods has shown that violations of expectations involve activity in inferior frontal regions, the caudate, and the nucleus \parencite[for reviews, see][]{salimpoor2015, trainor2016}. Finally, the use of computational methods has revealed that several modelling approaches can be used to successfully predict expectations \parencite[for a review, see][]{rohrmeier2012}. 

Naturally, some of the methodological limitations identified with regards to empirical aesthetics also apply to research on the relationship between expectation, emotion, and pleasure, which generally consists of gathering self-reports in response to relatively short \parencite{cheung2019} and often manipulated \parencite{gold2019, sauve2018, steinbeis2006} melodic and harmonic sequences.

\section{Music-evoked chills}    

One particular reaction to music, commonly referred to as chills, appears related to both aesthetics and expectation. There is little agreement on the exact definition of chills, on their physiological basis, or on their relationship to emotions. This motivated a systematic review of the literature on chills, presented in Chapter \ref{ch:2}, based on which we define chills for the purpose of the present thesis as a fleeting, pleasurable bodily sensation, sometimes accompanied by goosebumps, experienced when listening to specific musical passages. 

Music-evoked chills (MECs) are often considered to be a pleasurable response, and are thought to be stable, memorable, discrete, and observable, therefore addressing some of the limitations of self-reports and physiological measures. As a result, they have emerged as a convenient indicator of emotional and aesthetic experiences in research on responses to music. However, there is no consensus on how MECs relate to emotion, aesthetics, and pleasure, which motivates the need for further study. In fact, as revealed in Chapter \ref{ch:2}, we suspect that MECs are an optional but enhancing component of aesthetic and emotional responses, which therefore makes them unsuitable as the sole indicator of such responses.

Musical expectation has long been posited as a cause of MECs \parencite{harrison2014, huron2006, huron2010, juslin2013, juslin2008, mcdermott2012, mencke2019, pearce2012, salimpoor2011, sloboda1991}, but empirical investigations of this relationship suffer from theoretical and practical limitations, as does much of the wider research on MECs. Notably, there are significant shortcomings in the available literature regarding research design, adequacy of experimental variables, and empirical measures of MECs.

\section{Thesis overview}

The exact nature of the relationships between MECs, expectation, and affective and aesthetic responses remains unclear. The study of such relationships should aim to address some of the limitations which have historically affected research in empirical aesthetics, and stand to gain from the adoption of typically underused methodological approaches. Notably, computational methods applied to large collections of naturalistic stimuli are particularly well suited to the study of MECs, but have yet to be used, despite the success of similar approaches in research on music and emotion \parencite[e.g.,][]{eerola2011}.

Specifically, research on MECs should address the limitations discussed above by providing clarity on the relationship between MECs, aesthetic, and emotional responses, making use of naturalistic listening situations, and considering musical events as they dynamically unfold in time rather than as an aggregated whole. The present thesis seeks to do so by reviewing the suitability of using MECs in musical aesthetics and emotion research, detailing the construction of several large-scale, naturalistic datasets of pieces of music that elicit MECs, and using such datasets in a range of behavioural and computational studies. The presented research applies paradigms that address current gaps in the available knowledge on MECs, and particularly the widely hypothesised role of musical expectation and associated factors, while attempting to solve ongoing issues of reproducibility, generalisability, and ecological validity.

The context for the present work is presented as a systematic review of MECs in Chapter \ref{ch:2}. A longitudinal study of MECs is discussed in Chapter \ref{ch:3}, exploring the effects of musical content, stylistic preference, and familiarity. Computational studies of MECs are detailed in Chapters \ref{ch:4} and \ref{ch:5}, investigating expressed valence and musical expectation respectively. The thesis concludes with a discussion of the outcomes and implications of the present work in Chapter \ref{ch:6}.
