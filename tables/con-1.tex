\begin{table}[t!]
\centering
\scriptsize
\def\arraystretch{1.2}

\begin{threeparttable}
\caption{Summary of data collection steps}
\label{tab:con-1}

\begin{tabular*}{\textwidth}{
    >{\raggedright}p{0.22\textwidth}
    >{\raggedright}p{0.17\textwidth}
    >{\raggedright}p{0.16\textwidth}
    >{\raggedright\arraybackslash}p{0.345\textwidth}}

\hline

\textbf{Step} & \textbf{Data collection} & \textbf{Participants} & \textbf{Objective} \\ 

\hline
1. Online survey* & 
    Feb 2018 – \newline May 2018 &
    221 online (Qualtrics) & 
    Compile a list of tracks which \newline elicit MECs \\

\hline
2. Stimulus allocation & 
    Nov 2018 – \newline  Jun 2019 &
    33 online (psychTestR) & 
    Select unfamiliar tracks from \newline \emph{Step 1} for each participant \\

\hline
3. First lab session & 
    A few days \newline after \emph{Step 2} &
    30 in-person from \emph{Step 2} & 
    Collect continuous data as par- \newline ticipants listened to full tracks \\

\hline
4. Longitudinal phase & 
    Immediately after \emph{Step 3} &
    13 from \emph{Step 3} & 
    Increase familiarity with \newline previously unfamiliar tracks \\

\hline
5. Final lab session & 
    Two weeks \newline after \emph{Step 3} &
    Same as \emph{Step 4} & 
    Assess the effect of familiarity \newline on MECs \\
    
\hline

\end{tabular*}
\begin{tablenotes}
\small
\item Note. * The online survey was left running after May 2018 to build a dataset for the computational study detailed in Chapter \ref{ch:5}.
\end{tablenotes}
\end{threeparttable}
\end{table}