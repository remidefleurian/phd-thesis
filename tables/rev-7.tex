\begin{table}[t!]
\centering
\tiny
\def\arraystretch{1.2}

\begin{threeparttable}
\caption{Open issues, hypotheses, and suggested approaches}
\label{tab:rev-7}

\begin{tabular*}{\textwidth}{
    >{\raggedright}p{0.1\textwidth}
    >{\raggedright}p{0.24\textwidth}
    >{\raggedright\arraybackslash}p{0.555\textwidth}}

\hline

\textbf{Open issue} & \textbf{Hypothesis} & \textbf{Suggested approach} \\ 

\hline
Universality of chills & 
    MECs are experienced by the same proportion of the population, regardless of culture, but are dependent on enculturation. &
    Conduct a large-scale, cross-cultural survey of MECs, recording information about exposure to various musical cultures and genres. \\ 

\hline
Occurrence of piloerection &
    Piloerection occurs once MECs exceed an intensity threshold. &
    Record self-reported intensity of MECs and compare to measured piloerection. This might require further validation or development of piloerection sensors. \\
  
\hline  
Specificity of MECs &
    MECs exhibit different physiological and neural signatures than those of emotion or pleasure. &
    Compare responses with self-selected music that can elicit distinct experiences of MECs, emotion, and pleasure. If there is specificity, a classifier could be trained to distinguish unlabelled instances of MECs from emotional and pleasurable episodes without MECs. \\
    
\hline 
Acoustic and musical elicitors &
    The effect of acoustic elicitors on MECs is partially mediated by musical elicitors, and vice versa. &
    Compare extracted acoustic and musical features (using music information retrieval and/or manual annotation) around the onset of MECs (using a dataset such as the one provided in this review). Alternatively, systematically manipulate stimuli to independently vary the two types of elicitors and compare occurrences of MECs. \\
    
\hline
Familiarity &
    MECs are experienced more frequently as familiarity increases. &
    Use a longitudinal design to study the progress of the frequency of MECs when repeatedly exposed to previously unfamiliar and familiar music with the potential to elicit MECs. \\
    
\hline
MECs and attention &
    Attending to music increases the likelihood of MECs occurring, and MECs focus attention towards the eliciting music. &
    Assess the occurrence of MECs at rest and during a non-musical distractor task while listening to music. Fewer MECs should occur while distracted, and if they occur, they should impair performance on the task. \\
    
\hline
Psychological mechanisms &
    Exploratory animal and developmental research can help pinpoint the psychological and neural mechanisms underlying MECs. &
    Since brain stem reflex, musical expectation, and emotional contagion rely on different psychological and neural mechanisms, which might be more or less well developed in different species and at different developmental stages, exploring the prevalence of MECs in animals and individuals varying in developmental age could shed light on the mechanisms underlying MECs, and identify developmental trajectories. \\
    
\hline
Peak arousal &
    MECs can occur in response to peaks in arousal or pleasure, but might not always since several mechanisms drive the occurrence of MECs. &
    Record measures of physiological arousal for a large number of MECs to identify a threshold for peak arousal or subjective pleasure. MECs should happen every time this threshold is exceeded, but could also happen below this threshold if elicited by a different mechanism. \\
    
\hline
Musical expectation &
    MECs can occur in response to violations of expectation, but might not always since several mechanisms drive the occurrence of MECs. &
    Collect precise timing information for when MECs occur (or use the dataset provided in this review), and compare them to the output from a computational model of expectation. MECs should always occur for sufficiently strong violations of expectation but might occur elsewhere if elicited by a different mechanism. \\

\hline    
Evolutionary mechanisms* &
    MECs can occur via either peak arousal, contrastive valence, or the process of being moved. &
    Carefully prepare stimuli with the potential to elicit MECs via these three mechanisms, controlling for the others, and collect continuous measures (for instance, the two measures detailed above for peak arousal and expectation, and self-reports for being moved). Peaks for each measure should correspond to the onset of MECs for each targeted mechanism. \\

\hline    
Listener and context* &
    Susceptibility to MECs caused via different mechanisms is partly governed by individual differences, familiarity, and stylistic knowledge. &
    Using the approach detailed above, compare individual differences and personality correlates across participants who reported the most MECs for each mechanism. Include familiarity and stylistic knowledge for each piece of music as a random effect in a mixed effect model. \\

\hline    
Distinct types of MECs* &
    Different parameters and mechanisms cause different types of MECs, with distinct physiological and neural signatures. &
    Similarly, using the approach detailed above, compare physiological and neural correlates for MECs elicited via each mechanism. Alternatively, collect these measures along with qualitative descriptions of MECs to identify differences between different categories of MECs. \\
    
\hline

\end{tabular*}
\begin{tablenotes}
\footnotesize
\item Note. * Predictions derived from preliminary model of MECs.
\end{tablenotes}
\end{threeparttable}
\end{table}